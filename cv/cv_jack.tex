% LaTeX resume using res.cls
\documentclass[margin]{res}
\setlength{\textwidth}{5.1in} % set width of text portion
\usepackage{hyperref}
\usepackage{times}
\usepackage{color}
%%% HEADERS & FOOTERS
\usepackage{fancyhdr} 	% This should be set AFTER setting up the page geometry
\pagestyle{fancy} 		% options: empty , plain , fancy
\renewcommand{\headrulewidth}{0pt} % customise the layout...
\lhead{}\chead{}\rhead{}
\lfoot{}\cfoot{}\rfoot{\thepage}

\newcommand{\ignore}[1]{}

\begin{document}

% Center the name over the entire width of resume:
 \moveleft.5\hoffset\centerline{\LARGE \bf Yunhan (Jack) Jia}
% Draw a horizontal line the whole width of resume:
 \moveleft\hoffset\vbox{\hrule width\resumewidth height .5pt}\smallskip

\moveleft1\hoffset\vbox{
\begin{minipage}[t]{0.5\textwidth}
    \flushleft
    1195 Bordeaux Dr, Unit 16725\\ 
    Baidu USA, Sunnyvale, CA 94089\\
    Tel: (+1) 206-458-9830
\end{minipage}
\hfill
\begin{minipage}[t]{0.75\textwidth}
    %\flushright
    \flushleft
    Email: \href{mailto:jackjia@umich.edu}{\texttt{jackjia@umich.edu}}\\
    Homepage: \textbf{\url{http://www.jiayunhan.com}}\\
    Github: \href{https://github.com/jiayunhan}{\texttt{github.com/jiayuunhan}}\\
\end{minipage}
}

\begin{resume}


\section{RESEARCH \\ INTERESTS}
AI Security; Internet of Things security; Autonomous vehicle security;\\
\textit{I like to approach system security problems through systematic program analysis and design.}




\section{EDUCATION} 
University of Michigan, Ann Arbor \hfill 09/2013-04/2018 \\
PhD student in Computer Science and Engineering
\begin{itemize}
\item[-] Advisor: Professor \textbf{Z. Morley Mao}
\item[-] Thesis: Program Analysis based Approaches to Ensure Security and Safety of Emerging Software Platforms
\end{itemize}

Shanghai Jiaotong University, Shanghai, China \hfill 09/2009-06/2013 \\
Bachelor of Engineering in Software Engineering
\begin{itemize}
\item[-] Advisor: Professor \textbf{Haibo Chen}
\end{itemize}
%Rank: 1/91, Major GPA: 3.84/4.00, Cumulative GPA: 3.82/4.00

\section{WORK\\ EXPERIENCE}
\textbf{Senior Security Scientist} \hfill 04/02018-present\\
X-Lab, Baidu USA, US

\textbf{Threat Research Engineer Intern} \hfill 05/2017-08/2017 \\
Threat Research Team, Palo Alto Networks, US

\textbf{Research Intern} \hfill 06/2015-08/2015 \\
QoE Lab, T-Mobile, US

\textbf{Research Intern} \hfill 06/2014-08/2014 \\
QoE Lab, T-Mobile, US

\textbf{Intern} \hfill 08/2012-09/2012 \\
CSS Lab, Microsoft APGC


%%%%%%%%
\section{PUBLICATIONS}

Shichang Xu, \underline{Yunhan Jack Jia}, Z. Morley Mao, Subhabrata Sen,
Dissecting HAS VOD Services for Cellular: Performance, Root Causes and Best Practices, Proceedings of the 17th Internet Measurement Conference (\textbf{IMC}), 2017

Ding Zhao, Yaohui Guo, \underline{Yunhan Jack Jia}, 
TrafficNet: An Open Naturalistic Driving Scenario Library, Proceedings of the 20th IEEE International Conference on Intelligent Transportation System Conference (\textbf{ITSC}), 2017 \href{http://web.eecs.umich.edu/\~jackjia/material/trafficnet\_itsc17.pdf}{\color{blue}{[PDF]}}

\underline{Yunhan Jack Jia}, Ding Zhao, Qi Alfred Chen, Z. Morley Mao,
Towards Secure and Safe Appified Automated Vehicles, Proceedings of the 28th IEEE Intelligent Vehicles Symposium (\textbf{IV}), 2017 \href{https://arxiv.org/pdf/1702.06827.pdf}{\color{blue}{[PDF]}}

\underline{Yunhan Jack Jia}, Qi Alfred Chen, Shiqi Wang, Amir Rahmati, Earlence Fernandes, Z.Morley Mao, Atul Prakash, 
ContexIoT: Towards Providing Contextual Integrity to Appified IoT Platforms, Proceedings of the 24th Network and Distributed System Security Symposium (\textbf{NDSS}),2017 \href{http://web.eecs.umich.edu/\~jackjia/material/contexiot\_ndss17.pdf}{\color{blue}{[PDF]}}

\underline{Yunhan Jack Jia}, Qi Alfred Chen, Yikai Lin, Chao Kong, Z.Morley Mao, 
Open Port for Bob and Mallory: Open Port Usage in Android Apps and Security Implications, Proceedings of the 2nd IEEE European Symposium on Security and Privacy (\textbf{EuroS\&P}),2017 \href{http://web.eecs.umich.edu/\~jackjia/material/open\_euro17.pdf}{\color{blue}{[PDF]}}

Yuru Shao, Jason Ott, \underline{Yunhan Jack Jia}, Zhiyun Qian, and Z. Morley Mao,
The Misuse of Android Unix Domain Sockets and Security Implications, Proceedings of the 23th ACM Conference on Computer and Communications Security (\textbf{CCS}),2016 \href{http://web.eecs.umich.edu/\~jackjia/material/misuse\_ccs16.pdf}{\color{blue}{[PDF]}}

\underline{Yunhan Jack Jia}, Qi Alfred Chen, Z. Morley Mao, Jie Hui, Kranthi Sontineni, Alex Yoon, Samson Kwong, and Kevin Lau, Performance Characterization and Call Reliability Problem Diagnosis for Voice over LTE, Proceedings of the 21th ACM Annual International Conference on Mobile Computing and Networking (\textbf{MobiCom}),2015 \href{http://web.eecs.umich.edu/\~jackjia/material/performance\_mobicom15.pdf}{\color{blue}{[PDF]}}

Qi Alfred Chen, Zhiyun Qian, \underline{Yunhan Jack Jia}, Yuru Shao, and Z. Morley Mao, Static Detection of Packet Injection
Vulnerabilities -- A Case for Identifying Attacker-controlled Implicit Information Leaks, Proceedings of the 22nd ACM
Conference on Computer and Communications Security (\textbf{CCS}), 2015. \href{http://web.eecs.umich.edu/\~jackjia/material/static\_ccs15.pdf}{\color{blue}{[PDF]}}

\iffalse
\section{POSTERS/DEMOS}

\underline{Yunhan Jack Jia}, Qi Alfred Chen, and Z. Morley Mao, VoLTE Data Free-Ride Attack: A Case of Exploiting the Unprotected Voice Channel, Poster in 24th USENIX Security Symposium (Security), 2015.

Qi Alfred Chen, Zhiyun Qian, \underline{Yunhan Jack Jia}, Yuru Shao, and Z. Morley Mao, PacketGuardian: Systematic Detection of
Packet Injection Vulnerabilities using Precise Static Analysis, Poster in 24th USENIX Security Symposium (Security), 2015.

\underline{Yunhan Jack Jia}, Yihua Guo, Z. Morley Mao, Sung-Ju Lee, Collaborative DoS Attack against Cloud-hosted Web Services, Poster in 23th USENIX Security Symposium (Security), 2014. 

Qi Alfred Chen, \underline{Yunhan Jack Jia}, Zhiyun Qian and Z. Morley Mao, SystemLeakalyzer: Systematically Detecting System Side-Channels, Poster in 23th USENIX Security Symposium (\textbf{Security}), 2014.

Sanae Rosen, Hongyi Yao, Ashkan Nikravesh, \underline{Yunhan Jack Jia}, David Choffnes, Z Morley Mao, Demo: mapping global mobile performance trends with mobilyzer and mobiperf (\textbf{MobiSys}), 2014
\fi

\section{INVITED\\ TALKS}


The Cost of Learning from the Best: How Prior Knowledge Weakens the Security of Deep Neural Networks
\begin{itemize}
    \item Talk at \textbf{Blackhat Aisa}, Singapore, March 2019
\end{itemize}

Perception Deception: Physical Adversarial Attack Challenges and Tactics for DNN-Based Object Detection
\begin{itemize}
    \item Talk at \textbf{Blackhat Europe}, London, UK, December 2018
\end{itemize}

From Memory Safety to AI Security
\begin{itemize}
    \item Tech talk at Usenix Security Symposium (\textbf{Security}) BoF session, Baltimore, USA, August 2018 
\end{itemize}

Lessons Learnt from Securing Modern IoT and Autonomous Vehicle Platform
\begin{itemize}
    \item[-] Tech talk at Palo Alto Networks, Inc. Santa Clara, USA, July 2017
\end{itemize}


Towards Secure and Safe Appified Automated Vehicles
\begin{itemize}
    \item[-] 28th IEEE Intelligent Vehicles Symposium (\textbf{IV}), Redondo Beach, USA, June 2017
\end{itemize}

Open Port for Bob and Mallory: Open Port Usage in Android Apps and Security Implications
\begin{itemize}
    \item[-] 2nd IEEE European Symposium on Security and Privacy (\textbf{EuroS\&P}), Paris, France, April 2017
\end{itemize}

ContexIoT: Towards Providing Contextual Integrity to Appified IoT Platforms
\begin{itemize}
    \item[-] 24th Network and Distributed System Security Symposium (\textbf{NDSS}), San Diego, USA, February 2017
\end{itemize}

SmartPhone, SmartHome, and SmartCar: Lessons Learnt from Securing Modern Appified Platform
\begin{itemize}
\item[-] Invited talk at Shanghai Jiaotong University (\textbf{SJTU}), Shanghai, China, Janurary 2017
\end{itemize}

Performance Characterization and Call Reliability Problem Diagnosis for Voice over LTE
\begin{itemize}
\item[-] 21st  ACM Annual International Conference on Mobile Computing and Networking (\textbf{MobiCom}), Paris, France, September 2015
\end{itemize}

\section{PATENTS}
Monitoring Wireless Data Consumption, US Patent No. US20170005989 \href{https://www.google.com/patents/US20170005989}{\color{blue}{[Link]}}

Cross-Layer Link Failure Alerts, US Patent No. US20160065433 \href{https://www.google.com/patents/US20160065433}{\color{blue}{[Link]}}

Pathway-based Data Interruption Detection, US Patent No. US20160277952 \href{https://www.google.com/patents/US20160277952}{\color{blue}{[Link]}}




\ignore{
\section{RECENT\\ PROJECTS}

\textbf{Characterizing and } \hfill 08/2015-present 
\begin{itemize}
\item[-] Identify native layer interfaces that can be leveraged to access sensitive data or perform privileged operations, including the sysfs filesystem and local sockets.\itemsep -2pt
\item[-] Collect data from latest Android phones and analyze their SEAndroid policies to reveal insecure native interfaces that could be exploited.\itemsep -2pt
\item[-] Reverse engineer customizations made by vendors/carriers that introduce unprotected native interfaces and write exploits.
\end{itemize}

\textbf{Systematically Detecting Inconsistent Security Policy Enforcement in the Android Framework} \hfill 08/2014-08/2015
\begin{itemize}
\item[-] Designed and implemented Kratos, an automated tool that supports systematic detection of inconsistent security policy enforcement within the Android framework. \itemsep -2pt
\item[-] Applied Kratos to various versions of Android, discovered more than 10 zero-day vulnerabilities, and filed bug reports to Google. \itemsep -2pt
\item[-] A paper on this project will appear in NDSS 2016.
\end{itemize}
}

%\textbf{Intern} \hfill 06/2016-08/2016\\
%KNOX team, Samsung Research America

\section{HONORS\\ \& AWARDS}
Internet of Things (IoT) Technology Research Award, Google \hfill 2016\\
MobiCom Student Travel Grant \hfill 2015 \\
CSE Fellowship, University of Michigan \hfill 2014 \\
Rackham Travel Grant, University of Michigan \hfill 2014,2015 \\
USENIX Security Student Travel Grant, USENIX Association \hfill 2014,2015 \\
Excellent Participation in Research Program of Shanghai Jiao Tong University \hfill 2011 \\

\section{COMMUNITY\\ SERVICES}
PC member, ACM Workshop on Automotive Cybersecurity (\textbf{AutoSec}) 2019.
\end{resume}
\end{document}




